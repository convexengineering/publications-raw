\chapter{Extensibility of \gls{GP}}

Traditional engineering design optimization tools (eg. TASOPT) implement convergence loops that assume structure within a given design problem. 

The 'bag of constraints' form of the GP means that constraints can be added to the problem without 

\section{Improving fidelity: Adding a simple engine model to the SimPleAC}

The SimPleAC currently has an engine that weighs nothing and magically supplies unlimited power. This is obviously unphysical, and since one of the variables with the highest sensitivity is TSFC, requires refinement. 

If we think of an engine as an input-output system, we can determine how it would interact with the SimPleAC system, and create the appropriate variables. 

\begin{itemize}
\item Inputs: Fuel
\item Outputs: Power, noise, thrust, weight
\end{itemize}

\section{Expanding scope: Converting the SimPleAC to performance modeling form}

The SimPleAC that I have defined so far works well to demonstrate the capabilities of \gls{SP} in helping explore tradeoffs in engineering design. 

However, often in the design process, we will want to test the performance of a design in all

The performance modeling framework of GPkit was developed to facilitate the creation of vectorized variables and constraints to capture these effects. 