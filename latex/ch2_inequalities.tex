\chapter{Engineering inequalities and intuition, from equalities}

In this section, we will derive a simple aircraft problem that will serve as a demonstration of the intuitive structure of geometric programs. 

\section{Defining the aircraft design problem}

Aircraft epitomize the nature of complex engineering problems. The physical 
relations describing their motion are nonlinear, and all of their subsystems are 
coupled through the primary forces in flight (thrust, weight, lift, drag). 

\subsection{Functional definition}

The goal of the aircraft that we are designing will be to carry a given payload
over a distance while minimizing an objective function. 

\subsection{Component definition}

We can also think about the different components that make an aircraft function. 

For the basic example, the aircraft will need:
\begin{itemize}
	\item \textbf{A wing}: For generating lift. 
	\item \textbf{A fuselage}: For storing fuel. 
	\item \textbf{A payload}:
\end{itemize}

In the basic example, I choose not to model engines, and leave this as an exercise
to complete in Section~\ref{sec:engine}

\subsection{}

\subsection{Objective functions}

Objective functions are the way that a designer puts pressure on the variables 
in the constraints. 

In many design problems
formulated as \gls{GP}s, many different objectives will put pressure on design 
in the same direction

For example, an aircraft designed for fuel weight will look different than one 
that has been designed for total weight or payload-fuel consumption, but all of
these different objective will put a downward pressure on

\section{Limits of GP and Convexity} \label{sec:GPLimits}

Even with the demonstrated strengths of \gls{GP}'s in solving certain classes of problems, it is important to recognize that the mathematical framework has limits. 

It takes engineering intuition to recognize where and when the model needs to become non-log-convex for improved modeling. 

Even the addition of a single signomial constraint turns the problem from a \gls{GP} to a \gls{SP}, which means that the problems loses convexity and all of the mathematical guarantees associated with it. 

As an example, 

\subsection{Introducing fuel volume constraints to a GP aircraft model}

\subsection{What is the proper use case of the signomial equality?} 

Signomial equalities must be used as a last resort. The signomial equality is the only place where the feasibility set of individual \gls{GP}'s within a \gls{SP} solve are not guaranteed to be subsets of the feasibility set of the \gls{SP}. 