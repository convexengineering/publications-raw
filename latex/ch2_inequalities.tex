\chapter{Engineering inequalities and intuition, from equalities}

In this section, we will derive a simple aircraft problem that will serve as a demonstration of the intuitive structure of geometric programs. 

\section{Limits of GP and Convexity} \label{sec:GPLimits}

Even with the demonstrated strengths of \gls{GP}'s in solving certain classes of problems, it is important to recognize that the mathematical framework has limits. 

It takes engineering intuition to recognize where and when the model needs to become non-log-convex for improved modeling. 

Even the addition of a single signomial constraint turns the problem from a \gls{GP} to a \gls{SP}, which means that the problems loses convexity and all of the mathematical guarantees associated with it. 

As an example, 

\subsection{Introducing fuel volume constraints to a GP aircraft model}

\subsection{What is the proper use case of the signomial equality?} 

Signomial equalities must be used as a last resort. The signomial equality is the only place where the feasibility set of individual \gls{GP}'s within a \gls{SP} solve are not guaranteed to be subsets of the feasibility set of the \gls{SP}. 