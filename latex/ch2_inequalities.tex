\chapter{Engineering inequalities and intuition, from equalities}

In this section, we will derive a simple aircraft problem that will serve as 
a demonstration of the intuitive structure of geometric programs. 

\section{Defining the aircraft design problem}

Aircraft epitomize the nature of complex engineering problems. The physical 
relations describing their motion are nonlinear, and all of their subsystems are 
coupled through the primary forces in flight (thrust, weight, lift, drag). 

\subsection{Functional definition}

The goal of the aircraft that we are designing will be to carry a given payload
over a distance while minimizing an objective function. 

\subsection{Component definition}

We can also think about the different components that make an aircraft function. 

For the basic example, the aircraft will need:
\begin{itemize}
	\item \textbf{A wing}: For generating lift. 
	\item \textbf{A fuselage}: For storing fuel. 
	\item \textbf{A payload}:
\end{itemize}

In the basic example, I choose not to model engines, and leave this as an exercise
to complete in Section~\ref{sec:engine}

\subsection{Objective functions}

Objective functions are the way that a designer puts pressure on the variables 
in the constraints. 

In many design problems
formulated as \gls{GP}s, many different objectives will put pressure on design 
in the same direction

For example, an aircraft designed for fuel weight will look different than one 
that has been designed for total weight or payload-fuel consumption, but all of
these different objective will put a downward pressure on TSFC and wing weight. 

\section{Limits of GP and Convexity} \label{sec:GPLimits}

Even with the demonstrated strengths of \gls{GP}'s in solving certain classes of 
problems, it is important to recognize that the mathematical framework has limits. 
Certain constraints can only be expressed as signomial constraints, which are 
non-log-convex extensions of posynomial constraints. Even the addition of a single 
signomial constraint turns the problem from a \gls{GP} to a \gls{SP}, which means 
that the problems loses convexity and all of the mathematical guarantees associated with it. 
It takes engineering intuition to recognize where and when improved modeling is worth
the loss of the mathematical guarantees.

\subsection{Introducing fuel volume constraints to a GP aircraft model}

To demonstrate a model that requires the addition of signomial constraints, we will 
be adding a fuel volume model to SimPleAC, where fuel can be stored in the wing or
in the fuselage. The reason why this model is GP-incompatible is because of the 
following constraint which follows logically:

\begin{equation}
	V_{f_{avail}} \leq V_{f_{wing}} + V_{f_{fuse}}
	\label{vfavail}
\end{equation}

The fuel volume available must be less than the sum of the fuel volume available in the
wing and the fuselage. As such, it turns out that volumes that 'contain' free variables
can create signomial constraints. (One way around this is potentially creating fuel
fraction variables to denote how much fuel is stored in each volume, but I will not 
explore other potential parametrizations here.)

As such, we can continue to develop the model. We know that fuel weight is going 
to influence the lift required of the aircraft, so we must determine the weight of the fuel 
using a density parameter $\rho_{f}$. 

\begin{equation}
    V_f = \frac{W_f } {\rho_f g}
    \label{e:vf}
\end{equation}

We need a model of how much fuel volume there is in a wing. If we think about the total
volume in a wing, it is related linearly to its thickness ratio ($\tau$) and span ($b$), 
and to the square of its chord ($c$). 

\begin{equation}
	V_{f_{wing}} \propto \tau b c^2
	\label{e:vfwingpre}
\end{equation} 

In aerospace engineering, we like dimensionless quantities and planform areas, 
so we would like to express relation~\ref{e:vfwingpre} with respect to the planform area $S$
aspect ratio \AR and thickness ratio $\tau$ only, without having to define $b$ and $c$. 
Using the additional relation $S \propto b c$, we can express $V_{f_{wing}}$. 

\begin{equation}
	V_{f_{wing}} \propto \tau (\frac{AR}{S})^{0.5} (\frac{S}{b})^2 = 
		(\frac{AR}{S})^{0.5} \frac{S^2}{S \AR} = \frac{\sqrt{S}\tau}{\sqrt{AR}}
\end{equation}

\begin{equation}
    V_{f_{wing}}^2 \leq 0.0009 \frac{S \tau^2}{AR}
	\label{e:vfwing}
\end{equation}

(In the final form of the equation (\ref{e:vfwing}) I picked $0.009$ as the coefficient in front of the relation by tuning it after I had developed
the model, since this seems to give interesting results while doing trade studies.)

\subsection{What is the proper use case of the signomial equality?} 

Signomial equalities must be used as a last resort. The signomial equality is the only place 
where the feasibility set of individual \gls{GP}'s within a \gls{SP} solve are not guaranteed 
to be subsets of the feasibility set of the \gls{SP}. 